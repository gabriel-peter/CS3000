\documentclass[11pt]{article}

\newcommand{\yourname}{Gabriel Peter}
\newcommand{\yourcollaborators}{}

\newcommand\tab[1][1cm]{\hspace*{#1}}

\def\comments{0}

%format and packages

%\usepackage{algorithm, algorithmic}
\usepackage{algpseudocode}
\usepackage{amsmath, amssymb, amsthm}
\usepackage{enumerate}
\usepackage{enumitem}
\usepackage{framed}
\usepackage{verbatim}
\usepackage[margin=1.0in]{geometry}
\usepackage{microtype}
\usepackage{kpfonts}
\usepackage{palatino}
	\DeclareMathAlphabet{\mathtt}{OT1}{cmtt}{m}{n}
	\SetMathAlphabet{\mathtt}{bold}{OT1}{cmtt}{bx}{n}
	\DeclareMathAlphabet{\mathsf}{OT1}{cmss}{m}{n}
	\SetMathAlphabet{\mathsf}{bold}{OT1}{cmss}{bx}{n}
	\renewcommand*\ttdefault{cmtt}
	\renewcommand*\sfdefault{cmss}
	\renewcommand{\baselinestretch}{1.06}
\usepackage[usenames,dvipsnames]{xcolor}
\definecolor{DarkGreen}{rgb}{0.15,0.5,0.15}
\definecolor{DarkRed}{rgb}{0.6,0.2,0.2}
\definecolor{DarkBlue}{rgb}{0.2,0.2,0.6}
\definecolor{DarkPurple}{rgb}{0.4,0.2,0.4}
\usepackage[pdftex]{hyperref}
\hypersetup{
	linktocpage=true,
	colorlinks=true,				% false: boxed links; true: colored links
	linkcolor=DarkBlue,		% color of internal links
	citecolor=DarkBlue,	% color of links to bibliography
	urlcolor=DarkBlue,		% color of external links
}

\usepackage[boxruled,vlined,nofillcomment]{algorithm2e}
	\SetKwProg{Fn}{Function}{\string:}{}
	\SetKwFor{While}{While}{}{}
	\SetKwFor{For}{For}{}{}
	\SetKwIF{If}{ElseIf}{Else}{If}{:}{ElseIf}{Else}{:}
	\SetKw{Return}{Return}
	

%enclosure macros
\newcommand{\paren}[1]{\ensuremath{\left( {#1} \right)}}
\newcommand{\bracket}[1]{\ensuremath{\left\{ {#1} \right\}}}
\renewcommand{\sb}[1]{\ensuremath{\left[ {#1} \right\]}}
\newcommand{\ab}[1]{\ensuremath{\left\langle {#1} \right\rangle}}

%probability macros
\newcommand{\ex}[2]{{\ifx&#1& \mathbb{E} \else \underset{#1}{\mathbb{E}} \fi \left[#2\right]}}
\newcommand{\pr}[2]{{\ifx&#1& \mathbb{P} \else \underset{#1}{\mathbb{P}} \fi \left[#2\right]}}
\newcommand{\var}[2]{{\ifx&#1& \mathrm{Var} \else \underset{#1}{\mathrm{Var}} \fi \left[#2\right]}}

%useful CS macros
\newcommand{\poly}{\mathrm{poly}}
\newcommand{\polylog}{\mathrm{polylog}}
\newcommand{\zo}{\{0,1\}}
\newcommand{\pmo}{\{\pm1\}}
\newcommand{\getsr}{\gets_{\mbox{\tiny R}}}
\newcommand{\card}[1]{\left| #1 \right|}
\newcommand{\set}[1]{\left\{#1\right\}}
\newcommand{\negl}{\mathrm{negl}}
\newcommand{\eps}{\varepsilon}
\DeclareMathOperator*{\argmin}{arg\,min}
\DeclareMathOperator*{\argmax}{arg\,max}
\newcommand{\eqand}{\qquad \textrm{and} \qquad}
\newcommand{\ind}[1]{\mathbb{I}\{#1\}}
\newcommand{\sslash}{\ensuremath{\mathbin{/\mkern-3mu/}}}

%mathbb
\newcommand{\N}{\mathbb{N}}
\newcommand{\R}{\mathbb{R}}
\newcommand{\Z}{\mathbb{Z}}
%mathcal
\newcommand{\cA}{\mathcal{A}}
\newcommand{\cB}{\mathcal{B}}
\newcommand{\cC}{\mathcal{C}}
\newcommand{\cD}{\mathcal{D}}
\newcommand{\cE}{\mathcal{E}}
\newcommand{\cF}{\mathcal{F}}
\newcommand{\cL}{\mathcal{L}}
\newcommand{\cM}{\mathcal{M}}
\newcommand{\cO}{\mathcal{O}}
\newcommand{\cP}{\mathcal{P}}
\newcommand{\cQ}{\mathcal{Q}}
\newcommand{\cR}{\mathcal{R}}
\newcommand{\cS}{\mathcal{S}}
\newcommand{\cU}{\mathcal{U}}
\newcommand{\cV}{\mathcal{V}}
\newcommand{\cW}{\mathcal{W}}
\newcommand{\cX}{\mathcal{X}}
\newcommand{\cY}{\mathcal{Y}}
\newcommand{\cZ}{\mathcal{Z}}

%theorem macros
\newtheorem{thm}{Theorem}
\newtheorem{lem}[thm]{Lemma}
\newtheorem{fact}[thm]{Fact}
\newtheorem{clm}[thm]{Claim}
\newtheorem{rem}[thm]{Remark}
\newtheorem{coro}[thm]{Corollary}
\newtheorem{prop}[thm]{Proposition}
\newtheorem{conj}[thm]{Conjecture}

\theoremstyle{definition}
\newtheorem{defn}[thm]{Definition}


\newcommand{\instructor}{Drew van der Poel}
\newcommand{\hwnum}{1}
\newcommand{\hwdue}{Friday, May 21 at 11:59pm via \href{https://gradescope.com/courses/266585}{Gradescope}}

\theoremstyle{theorem}
\newtheorem{prob}{Problem}
\newtheorem{sol}{Solution}

\definecolor{cit}{rgb}{0.05,0.2,0.45} 
\newcommand{\solution}{\medskip\noindent{\color{DarkBlue}\textbf{Solution:}}}

\begin{document}
{\Large 
\begin{center}{CS3000: Algorithms \& Data} --- Summer I '21 --- \instructor \end{center}}
{\large
\vspace{10pt}
\noindent Homework~\hwnum \vspace{2pt}\\
Due~\hwdue}

\bigskip
{\large
\noindent Name: \yourname \vspace{2pt}\\ Collaborators: \yourcollaborators}

\vspace{15pt}
\begin{itemize}

\item Make sure to put your name on the first page.  If you are using the \LaTeX~template we provided, then you can make sure it appears by filling in the \texttt{yourname} command.

\item This assignment is due~\hwdue.  No late assignments will be accepted.  Make sure to submit something before the deadline.

\item Solutions must be typeset in \LaTeX.  If you need to draw any diagrams, you may draw them by hand as long as they are embedded in the PDF.  I recommend using the source file for this assignment to get started.

\item I encourage you to work with your classmates on the homework problems. \emph{If you do collaborate, you must write all solutions by yourself, in your own words.}  Do not submit anything you cannot explain.  Please list all your collaborators in your solution for each problem by filling in the \texttt{yourcollaborators} command.

\item Finding solutions to homework problems on the web, or by asking students not enrolled in the class is strictly forbidden.

\end{itemize}

\newpage

\begin{prob} Proof by Induction (8 points) \end{prob}


Prove the following statement by induction: For every $n \in \N$, $\sum_{i=1}^{n} i^2 = \frac{n(n+1)(2n+1)}{6}$

\solution \\

\emph{Base:} $n = 0$  such that $H(0) = 0^2 = \frac{(0)(0+1)(2(0)+1)}{6} = 0$ Pass! \\

\emph{Weak Induction:} \\

 $IH: H(k-1) =  \sum_{i=1}^{n-1} i^2 = \frac{(n-1)(n+1-1)(2(n-1)+1)}{6}  = \frac{(n)(n-1)(2n-1)}{6} = \frac{2n^3-3n^2+n}{6}$
	
 $H(k) = \sum_{i=1}^{n} i^2 =  \sum_{i=1}^{n-1} i^2  + n^2=  \frac{n(n-1)(2n-1)}{6} + n^2 = H(k-1) + n^2 =  \frac{2n^3+3n^2+n}{6} =  \frac{n(n+1)(2n+1)}{6}$ Pass!


\newpage
\begin{prob} Mystery Code (11 points) \end{prob}

You encounter the following mysterious piece of code.

\begin{algorithm}[H]
\caption{Mystery Function}
\Fn{$F(a,n)$}{
  \If{$n=0$}{\Return $(1,a)$}
  \Else{
    $b = 1$\\
    \For{$i$ from 1 to $2n$}{
      $b = b \cdot a$}
    $(u,v) \gets F(a, n-1)$\\
    \Return $(u \cdot b/a, v \cdot b \cdot a)$}
}
\end{algorithm}

\begin{enumerate}[label=(\alph*)]
\item \textbf{[3 points]} What are the results of $F(a,3)$, $F(a,4)$, and $F(a,5)$.  You do not need to justify your answers.

\solution

$F(a,3) = (a^{9}, a^{16})$\\
$F(a,4) = (a^{16}, a^{25})$\\
$F(a,5) = (a^{25}, a^{36})$\\
$F(a, n) = (a^{n^2}, a^{(n+1)^2})$

\item \textbf{[8 points]} What does the code do in general, when given input integer $n
  \ge 0$? Prove your assertion by induction on $n$.

\solution \\

Claim: $F(a, n) = (a^{n^2}, a^{(n+1)^2})$ where $n \ge 0$ \\
Base: $n = 0 \rightarrow F(a, 0) = (1,a)$ AND $(a^{0^2}, a^{(0+1)^2}) = (1, a)$ \\
Induction: We assume that $F(a, n-1) = (a^{(n-1)^2}, a^{(n)^2})$ is true for all $n \ge 0$\\

Provided from the pseudocode, we hit the else of our function and return:

\tab Thus, $F(a, n) = (u * b/a, v *ba)$ where $b = a^{2n}$ and $(u, v) = F(a, n-1) = (a^{(n-1)^2}, a^{(n)^2})$  \\
\tab $F(a, n) = (u * b/a, v *ba) =  (a^{(n-1)^2} * a^{2n}/a, a^{(n)^2} * a^{2n}*a)$ \\
\tab \tab $= (a^{n^2 - 2n +1} * a^{2n} * a^{-1}, a^{n^2} * a^{2n} * a^1)$ \\
\tab \tab $= (a^{n^2}, a^{(n+1)^2})$ via solving quadratic and is now equal to our claim! Fin.
\end{enumerate}

\newpage
\begin{prob} Stable Matching (14 points) \end{prob}


\begin{enumerate}[label=(\alph*)]
\item  \textbf{[6 points]} State the matching you obtain from running the Gale-Shapley algorithm on the following instance:

\begin{center}
\begin{tabular}{|l|c|c|c|}
\hline
hospital & 1 & 2 & 3  \\
\hline
$h_1$ & $d_2$ & $d_1$ & $d_3$   \\
$h_2$ & $d_2$ & $d_3$ & $d_1$   \\
$h_3$ & $d_1$ & $d_3$ & $d_2$   \\
\hline
\end{tabular}
\hspace{1in}
\begin{tabular}{|l|c|c|c|}
\hline
doctor & 1 & 2 & 3  \\
\hline
$d_1$ & $h_2$ & $h_3$ & $h_1$   \\
$d_2$ & $h_3$ & $h_1$ & $h_2$   \\
$d_3$ & $h_1$ & $h_3$ & $h_2$    \\
\hline
\end{tabular}
\end{center}

 Is the stable matching you found the only stable matching? If not, provide an example of another stable matching.

\solution

$ M = (h_1, d_2), (h_2, d_3), (h_3, d_1)$

\item \textbf{[8 points]} Given a set of preferences for $n$ doctors and $n$ hospitals, consider the stable matchings found via the following processes:

\begin{enumerate}
\item Run the standard Gale-Shapley algorithm with hospitals making offers to doctors. Let this matching be $M_1$.
\item Run Gale-Shapley again, but this time flip the roles of the hospitals and doctors in the algorithm, so that the doctors make offers to the hospitals. Let this matching be $M_2$.

\end{enumerate}

Prove the following claim:

\emph{If there is more than one stable matching, then $M_1 \neq M_2$.}

To do this, you may use the following terminology and Lemma 1.7 from the text. Hospital $h$ is a \emph{valid partner} of doctor $d$ if there is a stable matching the contains the pair $(h,d)$ (and vice versa). Doctor $d$ is the \emph{best/worst valid partner} of $h$ if every other valid partner is ranked lower/higher than $d$ in $h$'s preferences. When hospitals (doctors) propose in Gale-Shapley, each hospital (doctor) is paired with their best valid partner (Lemma 1.7). 

\solution \\

Proof by contradiction: Assume $M_1 = M_2$. \\

According to Lemma 1.7, "each hospital is paired with their best valid partner." \\




If  $M_1 = M_2$ then the best matchings for hospitals is equivalent to the best matchings for doctors. \\
The term "best" asserts that it was the only matching possible. \\
Thus, we can assume they reduce to a single matching due to their strict equivalence.
This is a contradiction as there are now not two stable matching's but one that occurs twice.

Therefore: $M_1 \neq M_2$, if there is more than one stable matching. \\




\end{enumerate}



\newpage
\begin{prob} Asymptotic Order of Growth (18 points) \end{prob}
\begin{enumerate}[label=(\alph*)]
\item \textbf{[10 points]} Rank the following functions in increasing order of asymptotic growth rate.  That is, find an ordering $f_1, f_2, \ldots, f_{10}$ of the functions so that $f_i = O(f_{i+1})$. No justification is required.

\begin{center}
\begin{tabular}{ccccc}
$n^3$ & $\sqrt{n}$ & $n!$ & $12^n$ & $\log_2 (n!)$  \\
$2^{4n}$ & $100 n^{3/2}$ & $10n$ & $2^{\log_3 n}$ & $\log_2^3 n$
\end{tabular}
\end{center}

\solution \\

$\sqrt{n}, 2^{\log_3 n}, 10n, \log_2^3 n, 100n^{3/2}, \log_2 (n!), n^3, 12^n, 2^{4n}, n!$

\item  \textbf{[8 points]} Suppose $f(n), g(n), h(n)$ are non-decreasing, non-negative functions. Decide whether you think the following statement is true or false and give a proof or a counterexample.

If $f(n) = \Omega(h(n))$ and $g(n) = O(h(n))$, then $f(n) = \Omega(g(n))$.

\solution \\
Given that $f(n) = \Omega(g(n)) \rightarrow f(n) \ge c * g(n)$ and $f(n) = O(g(n)) \rightarrow f(n) \le c * g(n)$ from Lecture 3/4 notes. \\

 $f(n) = \Omega(h(n)) \rightarrow f(n) \ge c * h(n)$ \\
 $g(n) = O(h(n)) \rightarrow g(n) \le c * h(n)$ \\
 Therefore, if we merge: $f(n) \ge c * h(n) \ge g(n)$ \\\\
 Transitively, $f(n) \ge g(n) \rightarrow f(n) = \Omega(g(n))$

\end{enumerate}



\end{document}
